\documentclass[12pt, a4paper]{report}
    \usepackage[T1]{fontenc}
    \usepackage{inputenc}
    \usepackage{pdfpages}
    \usepackage[toc]{glossaries}
    \makeglossaries
    \newglossaryentry{Rolling shutter}
    {
      name=Rolling shutter,
      description={When the camera needs to capture an image, it reads out pixels from the sensor a row at a time rather than capturing all pixel values at once}
    }
    
    \newglossaryentry{Balance wheel}
    {
      name=Balance wheel,
      description={is a wheel that regulates or stabilizes the motion of a mechanical watch and is responsible for the clock to run precisely}
    }
    
  \newglossaryentry{Time scale}
    {
      name=Time scale,
      description={is used to measure the rate deviation of a mechanical watch in different positions}
    }
    
    \newglossaryentry{SAD}
    {
      name=SAD,
      description={(sum of absolute differences) is a positive number, which results from the formation of the difference between two digital images. It serves as a measure of the difference between two images and is used in image processing and pattern recognition. It is obtained by subtracting the color values of the images pixel by pixel from each other and adding them up by amount}
    }
    
    \newglossaryentry{MMAL}
    {
      name=MMAL,
      description={(Multimedia Abstraction Layer) is a framework designed by Broadcom  to provide a host-side, simple and relatively low-level interface to multimedia components running on the Videocore IV GPU on the Raspberry Pi}
    }
    
    \newglossaryentry{Optical flow}
    {
      name=Optical flow,
      description={The optical flow of an image sequence is the vector field of the velocity projected into the image from visible points of the object in the reference system of the imaging optics}
    }
    
    \usepackage{booktabs}
    \usepackage{float}
    \usepackage{hyperref}
    \usepackage{mathtools}
    \hypersetup{
        colorlinks,
        citecolor=black,
        filecolor=black,
        linkcolor=black,
        urlcolor=black
    }
    
    \usepackage{tabularx}
        \newcolumntype{L}{>{\raggedright\arraybackslash}X}
    
    
    \begin{document}
    
    \begin{titlepage}
    
    \newcommand{\HRule}{\rule{\linewidth}{0.5mm}} % Defines a new command for the horizontal lines, change thickness here
    
    \center % Center everything on the page
    
    \textsc{\LARGE ZHAW School of Engineering}\\[1.5cm] % Name of your university/college
    \textsc{\Large Projektarbeit}\\[0.5cm] % Major heading such as course name
    \textsc{\large IT}\\[0.5cm] % Minor heading such as course title
    
    \HRule \\[0.4cm]
    { \huge \bfseries Title}\\[0.4cm] % Title of your document
    \HRule \\[1.5cm]
    
    
    \begin{minipage}{0.4\textwidth}
    \begin{flushleft} \large
    \emph{Authors:}\\
    Linda Helen \textsc{Boedi}  Valentin \textsc{Bossi} % Your name
    \end{flushleft}
    \end{minipage}
    ~
    \begin{minipage}{0.4\textwidth}
    \begin{flushright} \large
    \emph{Supervisor:} \\
     \textsc{Hans-Joachim Gelke} % Supervisor's Name
    \end{flushright}
    \end{minipage}\\[2cm]
    
    
    {\large \today}\\[2cm] 
    
    %\includegraphics{logo.png}\\[1cm] 
    
    \vfill 
    
    \end{titlepage}
    
    \pagenumbering{roman}
    \chapter*{Acknowledgement}
    \addcontentsline{toc}{chapter}{Acknowledgement}
    
    \chapter*{Affidavit}
     \addcontentsline{toc}{chapter}{Affidavit}
     
         
    \chapter*{Abstract}
     \addcontentsline{toc}{chapter}{Abstract}

\pagebreak
    \setcounter{secnumdepth}{5} 
    \setcounter{tocdepth}{5} 
    \pagenumbering{gobble}
    \tableofcontents
    \pagebreak
    \pagenumbering{arabic}
    
    \chapter{Conceptual formulation}
    
    \chapter{Motivation}
    
    \chapter{Fundamentals}
    \section {Competitor Analysis}
    \subsection{Optical versus acoustical measurement}
    To measure the frequency of the balance wheel of a mechanic clock there exist mainly two methods on the market, acoustical and optical measurement. The common used method is by analyzing the noises of the balance wheel. More expensive devices use both acoustical and optical feedback. In order to be able to classify the rate deviation well enough, it is usually measured in different positions of the watch.
    
    \begin{table}
     \centering
    \begin{tabularx}{\linewidth}{ |c||L|L|  }
     \hline
     \multicolumn{3}{|c|}{\textbf{Comparison acoustical and optical measurement}} \\
     \hline
     & \textbf{acoustical}  & \textbf{optical} \\\hline
      since   &  around 19th century (2)  & end of 20th century (2)\\ \hline
     accuracy &   0.1 s/d & 0.1 s/d\\  \hline
     advantages & most experience (exists for about 200 years)& measurement can be done anywhere\\  \hline
     disadvantages & background noises need to be filtered out& \\
     \hline
    \end{tabularx}
        \end{table}
    
    \subsubsection{Acoustical measurement}
    The first used method to get the frequency measurement of the balance wheel was by using a vibrograph was used. For every "tick" of the clock a line was drawn onto a ongoing strip of paper. Out of the distance between of these lines the frequency was calculated. (1) But as this method isn't the most accurate other techniques are used nowadays.  
    Modern watch timing machines use an oscillating quartz crystal as comparison for the frequency of the balance wheel (1). The noises of the balance wheel are recorded and amplified with a microphone whereat unwanted background noises need to be filtered out. 
    
    \subsubsection{Optical measurement}
    About hundred years later, optical watch timing machines joined the acoustical ones. There are barely devices on the market, which use only optical measurement, but several companies started to combine their acoustical with optical metering. One way to scale the frequency of the balance wheel optically is by using a laser. The beam will be periodically interrupted by the balance wheel and thus the frequency can be calculated (3). In this paper the frequency should be quantified optically by using image processing. There is hardly to none company out there, which uses the last technique for measuring the frequency.
    
    \subsection{Companies}
    \subsubsection{Witschi - WisioScope S}
    WisioScope S tests mechanical watches acoustically and optically. The measurement is done parallel and in this way is more accurate as both signals are used for the calculation of the frequency.
    The optical metering is done using a laser and lighting, a camera helps to adjust the watch properly. The costs of their product are: CHF 10450.-
    
    \bigskip
    
    \begin{tabularx}{\textwidth}{>{\bfseries}lX}
    Measurement gear deviation & \\\toprule
    scope & +/- 999.9 s/d \\\midrule
    resolution & 0.1 s/d\\\midrule
    price & CHF 10450.-\\\bottomrule
    \end{tabularx}
    
    \subsubsection{Lepsi - WatchScope/WatchAnalyzer}
    The WatchScope and WatchAnalyzer from Lepsi are especially for lovers and not really for production. Both products are Swiss technologies. Both devices only work with an acoustic input, with measurements of either a few seconds or up to 24 hours. All data can then be queried with the smartphone. WatchScope is the smaller, more convenient version. With the WatchAnalyzer you can easily measure the rate deviation of the watch in several positions. The prices for this product are CHF 369.00, resp. CHF 929.-
    
    \bigskip
    
    \begin{tabularx}{\textwidth}{>{\bfseries}lX}
    Measurement gear deviation & \\\toprule
    scope & +/- 1000 s/d \\\midrule
    resolution & 0.1 s/d\\\midrule
    price & CHF 369.-, resp. CHF 929.-\\\bottomrule
    \end{tabularx}
    
    
    \subsubsection{Greiner Vibrograf - Compact 900}
    The Compact 900 measures only the beat noises of the balance wheel. The price is CHF 4070.-
    
    \bigskip
    
    \begin{tabularx}{\textwidth}{>{\bfseries}lX}
    Measurement gear deviation & \\\toprule
    scope & +/- 1000 s/d \\\midrule
    resolution & 0.1 s/d\\\midrule
    price & CHF 4070.-\\\bottomrule
    \end{tabularx}
    
    \section{Pi camera module}
    
    The camera module of the Pi is basically a mobile phone camera module. Among other things, it uses a rolling shutter to take pictures. The pixel values of a frame are not captured completely at once, but are read line by line.
    
    However, the sensor is configured via the registers with the number of lines to be read and a corresponding time. The sensor reads the lines and pushes the data with the configured speed to the Raspberry PIPi.
    
    This keeps the readout time of each line constant. However, the CPU does not speak directly to the camera, but processing takes place on the GPU (VideoCore IV) of the Raspberry Pi, which operates its own real-time operating system (VCOS).
    
    \bigskip
    \noindent
    \begin{figure}
    \centering
    \includegraphics[scale=0.7]{Images/camera_architecture.png}
    
    \caption{Camera architecture. Readthedocs Picamera}
    \end{figure}
    
    
    
    \bigskip
    
    The illustration above illustrates the processing flow of a frame, with the associated steps explained in the following section.
    
    \begin{enumerate}
    \item The camera's sensor is configured and continuously streams frame lines to the GPU.
    \item The GPU builds complete frame buffers from these lines and performs the post-processing of these buffers.
    \item Meanwhile, myscript. py makes a capture call with the picamera on the CPU.
    \item The Picamera library uses the MMAL API to meet this requirement.
    \item The MMAL API sends a message via VCHI requesting a frame capture.
    \item The GPU then initiates a DMA transmission of the next full frame from its RAM portion to the CPU portion.
    \item Finally, the GPU sends a message via VCHI that the capture is complete.
    \item This causes an MMAL thread to trigger a callback in the Picamera library, which in turn retrieves the frame.
    \item Finally, write picamera calls to the output object provided by myscript. py.
    \end{enumerate}
    
    \subsection{Exposure time}
    
    The camera sensor detects how many photons hit the sensor elements, because the more impact, the more they increase their counter values. The sensor can perform exactly two operations; reset a set of elements or read a set of elements.
    
    \subsubsection{Minimal exposure time}
    
    Reading out a series of elements takes a certain amount of time, thus there is a limit to the minimum exposure time. Assuming one has 500 lines on a sensor and reading each line takes at least 20ns, then it will take at least $500*20 \text{ns} = 10 \text{ms}$ to read a full screen. 
    
    \subsubsection{Maximum framerate}
    
    The frame rate is the number of frames the camera can capture per second. The exposure time determines the maximum number of images that can be taken in a given time. Assuming it takes 10ms to read a complete image, then no more than $\frac{1 \text{s} }{10 \text{ms}} = \frac{1\text{s}}{0.01\text{s}} = 100 $ images in one second. 
    
    \bigskip
    
    So it is valid: 
    \begin{displaymath}
    \frac{1\text{s}}{\text{min exposure time in s}} = \text{max framerate in
    fps.} 
    \end{displaymath}
    The lower the minimum exposure time, the higher the maximum frame rate and vice versa.
    
    \subsubsection{Maximum exposure time}
    Um die Belichtungszeit zu maximieren, muss die Framerate reduziert werden. 
    
    Es gilt :
    \begin{displaymath}
    \frac{1\text{s}}{\text{min framerate in fps}} = \text{max exposure time
    in s}
    \end{displaymath} 
    
    \subsection{Hardware limits}
    
    \begin{itemize}
    \item The maximum resolution for MJPEG recording is partially dependent on the GPU memory.
    \item The maximum horizontal resolution for the standard H264 recording is 1920 (this is a limitation of the H264 block in the GPU).
    \item The maximum frame rate of the camera depends on several factors. With overclocking 120fps can be achieved, but 90fps is the maximum supported frame rate.
    
    \end{itemize}
    
    
    \section{Mechanical watches}  
    \subsection{Operation of a mechanical watch}  
    If a mechanical watch is wound up over the crown, the locking wheel (3) is moved by the winding shaft (1) and winding wheels (2). It turns the spring core (4) and pulls the tension spring in the mainspring barrel (5). The spring transmits the stored energy via the external toothed mainspring barrel (5) to the minute wheel (6), third wheel (7) and fourth wheel (8).
    \newline
    \noindent
    \begin{figure}
    \centering
    \includegraphics[scale=0.4]{Images/Funktionsweise-Uhrwerk.jpg}
    
    \caption{Operation of a mechanical watch}
    \end{figure}
The so-called escapement ensures that the gear train runs at the correct speed. The fourth wheel (8) drives the escapement wheel (9); it gives an impulse to the armature (10) which it passes on to the balance (11), after which the armature, which moves back and forth like a seesaw, blocks the escapement wheel. The balance wheel rotates, but is then retracted by the hairspring (12). It moves the armature back, which now releases the armature wheel again a little bit, which gives the armature with its rotation the next impulse. Due to the sudden braking and acceleration of the escapement wheel, the second hand of a mechanical watch also moves stepwise. [6]
    \newline
    \noindent
    \begin{figure}
    \centering
    \includegraphics[scale=0.4]{Images/Hemmung-Teil1.jpg}
    
    \caption{Escapement and balance wheel}
    \end{figure}
\bigskip
The balance wheel is the heart of the oscillation system. It generates a time-defined movement, which in turn is transferred to the gear train and passed on to the watch hands. 
The occurring error, if the balance wheel does not run accurate, is called gang deviation. The deviation can be caused by different external and internal matters. Internal motives often are wear, resp. erosion or dirt. Whereas there are a lot more external causes as changes in temperature, air pressure or magnetism.
    
    \subsection{Number of impacts}
    Since the rate deviation has been calculated in seconds per day and the number of strokes per hour, i. e. the number of audible impulses (beats) of a two-armed armature of a mechanical watch per hour, is known, these two quantities must first be brought to a unit. The relationship between the beat number (n*) and frequency (f) is as follows: 
    
    \begin{displaymath}
    n* = 2f*3600
     \end{displaymath}
     and therefore:
     \begin{displaymath}
      f = \frac{n*}{7200}
     \end{displaymath}
     
    
     \begin{table}
     \centering
    \begin{tabularx}{\linewidth}{ |L|L|L|L|L|  }
     \hline
     \textbf{Number of impacts} [1/h] &  \textbf{Impact duration} [s]& \textbf{Oscillation number}  [1/h]& \textbf{Period duration} [s]& \textbf{Frequency} [Hz]\\\hline
      18'000   &  0.200  & 9'000 & 0.400 & 2.50\\ \hline
     19'800 &  0.182 & 9'900 & 0.364 & 2.74\\  \hline
     21'600 &  0.166 & 10'800 & 0.333 & 3.00\\  \hline
    28'800 &  0.125 & 14'400 & 0.250 & 4.00\\  \hline
    36'000 &  0.100 & 18'000 & 0.200 & 5.00\\  \hline
    \end{tabularx}
       \caption{  Number of strokes, period duration and frequency of the balance of automatic wristwatches}
        \end{table}
        
        \subsection{Test watch}
    At first some general information to the test watch, which was used.
    
    The test watch is a watch of the caliber ETA C01.211 it is a chronograph, with a diameter of 31 centimeters. It has a beat rate of 21600 beats per hour and therefore a frequency of 3 Hz. Additionally the watch has a power reserve of 43 hours. (5)
    
    In order to obtain an exact reference value of the frequency of the balance wheel of the test watch, an acoustic measurement was carried out with a professional measuring device, the Greiner Vibrograf - Compact 900, and the rate deviation was determined. 
    
    In order to calculate the frequency from the gear deviation of the balance wheel, first some transformations had to be carried out. 

     
    
    \chapter {Experimental approaches}
    
    \section {Evaluation of possible technologies}
    \subsection{Frame analysis versus given GPU motion vectors}
    One approach is to compare each image with the subsequent image and to mark the displacements with motion vectors. This would mean a considerable computational effort.
    Such motion vectors are already made by the GPU, in order to compress videos in the H.264 codec.
    
    \subsection{Frame analysis via OpenCV}
    With the library OpenCV it is possible to determine the optical flow of an image sequence. 
    
    The optical flow represents the vector field of the velocities of the points of an image sequence. It is a useful representation of motion information in image processing. The local optical flow is an estimate of patterns in an image in the vicinity of a viewed pixel. 
    
    One possibility would be to follow the course of a velocity vector and thereby calculate a period of the frequency of the balance wheel. The calculation of the flow at selected points is also called feature point tracking. Another approach would be to calculate the frequency based on the velocity obtained from the optical flow, since it is more or less a circular movement. 
    
    One difficulty with this procedure is that one has to commit to certain pixels that one examines. This can cause problems if the clock is placed differently, because wrong points (worst-case scenario: points in which movement never takes place) can cause problems for the calculation.  
    
    The OpenCV library uses the Lucas-Kanade method to calculate the optical flow. The Lucas-Kanade method also assumes that the flow in the local environment of the pixel for which the flow is intended is the same. So more or less a whole set of pixels is considered. The flow can thus be determined by calculating the derivatives (= gradients). 
    
    One characteristic of this method is that it does not provide a dense flow, i. e. the flow information disappears quickly with the distance from the edges of the image. The advantage of the method is its relative robustness against noise and smaller defects in the image.
    
    \subsection{Motion vectors from high level library picamera}
    The picamera library is capable of storing all motion vector estimations of the h.264 encoder in an object. This object with all motion vectors for all frames can be later accessed and used. The motion data is at the level of macro-blocks where the values are 4-bytes long and consist of a signed 1-byte x vector, a signed 1-byte y vector and an unsigned 2-byte SAD (Sum of Absolute Differences) value. All motion data values can be easily accessed frame by frame. (2) 
    
    The sum of absolute differences is a positive number resulting from the formation of the difference between two digital images. It serves as a measure of the difference between two images.
    
    The SAD is obtained by subtracting the color values of the images pixel by pixel from each other and adding them up by amount. (3)
    
    The advantage of the Picamera library is that compared to the possibilities of the OpenCV library, the information about the motion vectors is available faster, since this library already has an internal object (motiondata), which stores all necessary information as well as the SAD value. 
    
    \section{Animation of motion vectors}
    After deciding which library to use, some experimental sessions took place to get a better understanding of the so called motion data values or motion vectors. 
    With the picamera library it is easy to access these motion data values so the hard part wasn't to get the values but to understand how they are constructed. 
    The motion data object consists of multiple values; a signed 1-byte x vector, a signed 1-byte y vector and and unsigned 2-byte SAD value for each macro-block of a frame.
    
    In a first step all SAD values were animated with colors to get a better understanding of how good the information they keep is and how as well as if they can be used to calculate the frequency of the movement of the balance wheel.
    
    \noindent
    \begin{figure}
    \centering
    \includegraphics[scale=0.6]{Images/animation_sad.png}
    
    \caption{Animation of the SAD values, the lighter the color the higher the value of the SAD}
    \end{figure}
     
    Further the x and y vector were analyzed and also animated. But as those vectors itself are hard to use and only give partial information (y-vector: up or down, x-vector: right or left), the hypotenuses were calculated with Pythagoras' theorem and displayed similar to the SAD values. The hypotenuses give information about the amount as well as the direction of the motion.
    
    \noindent
    \begin{figure}
    \centering
    \includegraphics[scale=0.6]{Images/animation_hypotenuse.png}
    
    \caption{Animation of the hypotenuses, the lighter the color the higher the value of the hypotenuse}
    \end{figure}
    
    Another experimental approach was displaying the motion vectors as arrows directly in the video. This approach was helpful to get a better understanding under which conditions the best result of the motion vectors is reached.
    \pagebreak
    
    \subsection{Motion vectors visualized in a frame}
    Codecvisa was used to display the motion vectors of a frame.
    The result is shown in Figure 3.
 
    \noindent
    \begin{figure}
    \centering
    \includegraphics[scale=0.5]{Images/motion_vectors.png}
    
    \caption{Motion vectors drawn in one frame}
    \end{figure}
    
    Another approach used Excel was to determine whether a motion vector from one frame is the predecessor of the motion vector in the next frame.
    The hope was that a moving pixel could be followed and therefore the location of the pixel would always have been known.
    The result was sobering because one frame and motion vector cannot be associated with another frame and motion vector.
    This experiment looked at the idea of the optical flux with the movement vectors of the MMAL (Multi-Media Abstraction Layer).
    
    \noindent
    \begin{figure}
    \centering
    \includegraphics[scale=0.7]{Images/optical_flow_basic1.jpg}
    
    \caption{It shows a ball moving in 5 consecutive frames. The arrow shows its displacement vector.}
    \end{figure}
    
    \newpage
    
    \section{Picamera library}
    The Picamera library is an open source project in the programming language Python. The library is an interface to the MMAL ---written in C--- which in turn is an interface to the firmware of the GPU.
    This means that the camera can be accessed using a simple programming language such as Python.
    
    \subsection {Picamera settings}
    Testing with different settings has shown that the results can be improved. For example, the black and white shots are the better choice. Reducing the shutter speeds can reduce motion blur, which also contributes to a better result.
    The consequence of the shorter exposure time is that the light source must be very strong and close to the subject, otherwise the image is too dark and has a negative effect on the result.
    
    
    \section{Crude calculation of frequency}
    With the help of the motion vectors and the corresponding time stamps per frame, a rough calculation of the frequency can be made. All x, y or SAD values contained in the image are summed up. If the sum of all values is small, it is very likely that the balance wheel is at a standstill before turning back again. For small values, the time stamp is registered and the difference to the next standstill is calculated. Since the balance wheel triggers half a oscillation, it has moved from one stop to the next half point. 
    In order to roughly calculate the frequency, an Excel sheet was created in which all the accumulated x- and y-values were entered and the standstills of the balance wheels were determined by eye. The half-oscillations and the frequency were calculated from the time intervals of the standstills. Afterwards, the average of all calculated frequencies was determined and it was recognized that the results with this method are good enough to determine the frequency and thus the gear deviation. 
    
    \noindent
    \begin{figure}
    \centering
    \includegraphics[scale=0.7]{Images/excel_sheet_numbers.png}
    
    \caption{Excel sheet with summed x-, y-values, timestamps and calculated frequency}
    \end{figure}

    \noindent
    \begin{figure}
    \centering
    \includegraphics[scale=0.4]{Images/excel_sheet_graph.png}
    
    \caption{Illustration of summed x-(blue), y-values (red) over time. The x-axis is the time and the y-axis are the x- and y-values.}
    \end{figure}
    
    \pagebreak

    \section{Verification of accuracy}
    The calculated frequency is far too imprecise. In theory, however, the calculation should become more precise after a certain period of time and should even be accurate to the microsecond.
    But this has never been achieved. For this reason, a reliable artificial clock was used in the form of an LED that is clocked by a 50 Mhz quartz and is therefore sufficiently accurate.
   
   \begin{table}
      \centering
        \begin{tabularx}{\linewidth}{ |L|L|L|L|L|L|  }
        \hline
        \textbf{Duration} &  \textbf{run 1} &  \textbf{run 2} &  \textbf{run 3} &  \textbf{run 4} &  \textbf{run 5} \\ \hline
        10 sec        & 166591.5                 & 166585.0     & 166397.2     & 165655.4      & 166585.0      \\ \hline
        20 sec     & 166679.0                 & 166117.1      & 166675.0    & 166675.0      & 166581.9  \\ \hline
        40 sec      & 166486.4                 & 166673.1    & 166675.0     & 166440.3    & 166673.1    \\ \hline
        60 sec      & 166673.7                 & 166672.4	   & 166673.7	  & 166672.4	  & 166673.7  \\ \hline
        300 sec      & 166665.2                 & 166671.7	   & 166640.4	  & 166665.3	  & 166640.4      \\ \hline
        600 sec      & 166668.2                 & 166649.6	   & 166668.2	  & 166655.8	  & 166668.2      \\ \hline
        900 sec      & 166667.2                 & 166667.3	   & 166638.4	  & 166638.4	  & 166667.2      \\ \hline
        1200 sec      & 166660.4                 & 166668.3	   & 166669.7	  & 166668.2	  & 166668.2      \\ \hline
        1500 sec      & 166668.8                 & 166650.3	   & 166650.3	  & 166668.8	  & 166668.8      \\ \hline
        1800 sec      & 166652.8                 & 166668.2	   & 166669.2	  & 166668.2	  & 166669.2      \\ \hline
        2100 sec      & 166668.6                 & 166668.6	   & 166668.6	  & 166655.4	  & 166654.5      \\ \hline
        2400 sec      & 166668.1                 & 166652.7	   & 166656.5	  & 166668.1	  & 166668.1        \\ \hline
    \end{tabularx}
     \caption{}
    \end{table}
     
    \chapter{Calculation of the frequency}
    
    \section{Implementation of the algorithms}
    \subsection{Calculation of the frequency by counting minima with upper limit}
    
    In a next step, the results of the rough calculation from the Excel sheet were transferred into a Python program. 
    
    The greatest difficulty proved to be the detection of the minimia, which characterize the standstill of the balance wheel. These were relatively easy visible to the unaided eye, but in an automatic calculation, appropriate conditions had to be set in order to take into account the correct values and correct the inaccuracies caused by the noise. The setting of an upper limit for the minima proved to be a suitable condition. All minima below this limit are therefore taken into account. 
    
    Furthermore, the calculation of the frequency in the Python program has been further simplified by not calculating each period individually, but by selecting the first and the last minimum over the whole period of time, in which it was recorded. This time span is then calculated by half of the number of minima minus 1. Half, because every half of the period the balance wheel stops and minus 1, because otherwise a half-period would be taken into account too much. This approach reduces the number of rounding errors, resulting in higher accuracy. After this calculation, the period duration T is now available, from which the frequency f can be calculated quite simply: 
    
     \begin{displaymath}
      f = \frac{1}{T}
     \end{displaymath}
     
  \subsection{Calculation of the frequency by counting minima with stepsize}
    
    \chapter{Results}
    
    \section{Measurements}
    Reference measurements were performed on the test watch on three days. Two measurements were carried out with the Greiner Compact 900 and one with the Wisio Scope meter. Calculations using the Greiner Compact diverge greatly from those taken with the Wisio Scope. In the next section, this divergence and its causes will be discussed in more detail. 
\bigskip
\newline
As the test watch was in a distinctly colder environment (about 5 degrees Celsius) shortly before the Greiner Compact meter was used to measure the reference values, it is highly likely that the small metal spring has contracted and the watch ran faster as a result. Before the calculation with the Wisio Scope time scale, the test watch was in an ambient temperature of about 25 degrees Celsius for a long time. A difference of only 5 degrees Celsius can affect the accuracy of the watch. As the reference measurements are based on a difference of about 20 degrees Celsius, the above-mentioned difference occurs with the measured values obtained.
The optical measurements of the picamera were carried out on an average at about 25 degrees Celsius, therefore the reference values determined by the Wisio Scope time scale are best suited for the comparison.
\bigskip
\newline
    All measurements were taken in different positions of the watch with different length
    of recordings whereas each position was recorded in 10, 20, 40 and 80 minutes.
\bigskip
\newline
    Figure 5 and Table 3 show the deviation per day based on the following calculation:
    Taken the first timestamp (t1) of a stillstand and taken the last timestamp (t2) of a stillstand, the counted number of stillstands (s1), number of strokes per hour (s2) (21600):
    \bigskip
    
    \(t2-t1 = \triangle t\)
    
    \(\triangle t / s = f*2\)
    
    \((f*2*s2)-3600=\text{deviation per hour}\)
    
    \noindent
    \begin{figure}
        \includegraphics[scale=0.5]{Images/variations_per_day.png}
    
    \caption{ Variation of different time and position recordings}
    \end{figure}    
    
    \begin{table}
      \centering
        \begin{tabularx}{\linewidth}{ |L|L|L|L|L|L|L|L|  }
        \hline
        {\fontsize{10}{12}\selectfont \textbf{Positions}} &{\fontsize{9}{10}\selectfont \textbf{Greiner Compact 900}} & {\fontsize{10}{12}\selectfont \textbf{Wisio Scope}} & {\fontsize{10}{12}\selectfont \textbf{10 min}} &  {\fontsize{10}{12}\selectfont \textbf{20 min}} &  {\fontsize{10}{12}\selectfont \textbf{40 min}} & {\fontsize{10}{12}\selectfont  \textbf{80 min}} \\ \hline
        up        & -13                 &      -31  & -48.7     & -48.0     & -47.6      & -47.0      \\ \hline
        right     & -30                 &      -64  & -58.3      & -65.5    & -54.4      & -49.4      \\ \hline
        down      & -20               &  -46  & -122.2    & -122.2     & -125.4    & -121.4      \\ \hline
        left      & -14                & -34 & -87.0	   & -87.1	  & -87.1	  & -83.1      \\ \hline
        \end{tabularx}
          \caption{ variations per day in seconds} 
    \end{table}
    
    \noindent
    \begin{figure}
    \includegraphics[scale=0.5]{Images/smallest_variation.png}
    
    \caption{ Variation of different time and position recordings}
    \end{figure} 
    
    \section{Calculation of gang deviation from frequency}
    
    \section{Evaluation}
    
    
    \glsaddall
    \printglossaries
    
    \begin{thebibliography}{9}
    \bigskip
    \bibitem[Uhrenwiki]{Uhrenwiki} 
    (1) https://www.uhren-wiki.net/index.php?title=Zeitwaage
    \bibitem[ncsli]{ncsli} https://www.ncsli.org/c/f/p11/286.314.pdf
    \bibitem[chronoscop]{chronoscop} http://www.chronoskop.com/index.php/chronoskop-zeitwaagen-von-prelislisit-timegraphers/
    \bibitem[Picamera]
    (2) picamera readthedocs
    \bibitem[SAD]
    (3) Wikipedia: Sum of absolute differences
    \bibitem[Schlagzahl]
    (4) Wikipedia: Schlagzahl
    \bibitem[Caliber Test watch]
    (5) http://watchbase.com/eta/caliber/c01-211
    \bibitem[Das mechanische Uhrwerk]
    (6) https://www.watchtime.net/uhren-wissen/das-mechanische-uhrwerk/
    \end{thebibliography}
    
    \pagebreak
    
    \listoffigures
    \bigskip
    
    
    
    \end{document}